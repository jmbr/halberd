\documentclass[a4paper]{book}
\usepackage{epsfig}
\usepackage[english]{babel}
\usepackage[latin1]{inputenc}
\begin{document}

\title{\Large \bf halberd user's guide}
\author{Juan M. Bello Rivas}
\date{}
\maketitle

\tableofcontents

% 1. Explain the problem being solved.
% 2. Present the concepts, not just the features.
% 3. Give'em more than they deserve.
% 4. Make it enjoyable to read.

\chapter{Introduction}

\section{Motivation}

In order to cope with heavy loads, web site administrators often install load
balancer devices.  These machines hide several web servers behind a virtual IP
(VIP from here on) taking HTTP requests and redirecting them to the real web
servers according to several criteria in order to share the traffic between all
the real servers.

There are a few ways to map the servers behind the VIP and to reach them
individually.

halberd is a tool for discovering HTTP load balancers. It is useful in testing
load balancer configurations as well as auditing web applications.

Identifying and being able to reach all real servers individually (effectively
bypassing the load balancer) is very important for an attacker trying to break
into a site because it is often the case that there are configuration
differences ranging from slight:

\begin{itemize}
  \item server software versions,
  \item server modules
\end{itemize}

to the extreme:

\begin{itemize}
  \item different platforms
  \item server software.
\end{itemize}

For an attacker this information is crucial because he might find vulnerable
configurations that otherwise (without mapping the real servers) could have
gone unnoticed.

But someone trying to break into a website doesn't have server software as its
only target. He will try to subvert dynamic server pages in several ways.  By
identifying all the real servers and scanning them individually for
vulnerabilities he might find bugs affecting only one or a few of the web
servers. Even if all machines are running the same server software halberd can
detect them and this can lead to more through vulnerability scans on the
application level.

\section{Concepts}

TODO Explain what a VIP is, time differences, clues, etc.

\chapter{Installation}

TODO Write about prerrequisites, tested platforms, etc.

\chapter{Operation}

\section{Command-line options}

TODO Explain all the options and their annoyances.

\section{CLU file format}

\end{document}

% vim: ts=2 sw=2 et ft=tex
